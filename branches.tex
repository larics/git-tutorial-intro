\section{Working with branches}

\begin{frame}
	\frametitle{Branches in git}
	
	\begin{enumerate}
		\item Branches are used to develop features in isolation from each other
		\item Git is specifically designed for efficient work with branches
		\item A branch is simply a pointer to a commit
		\item The default branch name is \alert{master}
	\end{enumerate}	
	
	
\end{frame}

% -----------------------------------------------------------------------------

\begin{frame}[fragile]
	\frametitle{Creating a branch}
	
	\begin{block}{When do I need a branch?}
	Generally, every time you start working on a new feature, or any significant change to your code.
	\end{block}
	
	Creating a branch:
	\begin{minted}{console}
	> git branch <yourname>-devel
	\end{minted}
	
	Listing local branches:
	\begin{minted}{console}
	> git branch -v
	\end{minted}	
	
	Switching between branches:
	\begin{minted}{console}
	> git checkout <yourname>-devel
	\end{minted}
	
	% TODO: add figures!
	
\end{frame}

% -----------------------------------------------------------------------------

\begin{frame}[fragile]
	\frametitle{Working on a branch}
	
	\begin{block}{Task}
	Edit some files. Create some files. Commit your work.
	\end{block}	
	
	Switching back and forth between branches changes the files on your disk!
	
	After you are finished working on the new feature, it's time to merge it back into the master branch.
	\begin{minted}{console}
	> git checkout master
	> git merge <yourname>-devel
	> git branch -d <yourname>-devel
	\end{minted}
	It's very good practice to clean up after yourself and delete unused branches.
	
	Merge conflicts are handled in the same way as discussed before.
\end{frame}

% -----------------------------------------------------------------------------

\begin{frame}[fragile]
	\frametitle{Remote branches}
	
	\begin{block}{Task}
	Create a new branch named \texttt{<yourname>-remtest} and switch to it. Make some changes.
	\end{block}	

	We need to explicitly push the local branch to the remote repository:
	\begin{minted}{console}
	> git push -u origin <yourname>-remtest
	\end{minted}	
	
\end{frame}

% -----------------------------------------------------------------------------

\begin{frame}[fragile]
	\frametitle{Pull requests}
	
	\begin{block}{Task}
	Look for your branch in the web interface of the GitHub repo. Create a pull request for your branch. Have your partner review and merge the pull request. Delete the branch after it has been merged.
	\end{block}	

	\begin{block}{Code review}
	Pull requests are an efficient and transparent code review mechanism. Code review is good. Pull requests are good. Use pull requests :)
	\end{block}
\end{frame}

% -----------------------------------------------------------------------------

\begin{frame}[fragile]
	\frametitle{Working with multiple remotes}
	
	\begin{block}{Why would I need multiple remotes?}
	Multiple remotes allow us to get code changes directly from other developers, without going through the "main" server. A typical use-case is when working in the field, with no internet access. All that is required is an ssh connection to the other machine (e.g. via ad-hoc networking) and a git user with appropriate permissions set up on that machine. 
	\end{block}	

	Listing and adding remotes:
	\begin{minted}{console}
	> git remote add <alias> git@<hostname>:<path to repo>	
	> git remote -v	
	\end{minted}
	
	We can now work with the new remote in the same way as with origin (except for pushing!), e.g.:
	\begin{minted}{console}
	> git fetch <alias>
	> git merge <alias>/<branch>
	\end{minted}
	
\end{frame}

% -----------------------------------------------------------------------------
