\section{Keeping track of your code}

\begin{frame}

\frametitle{Version control system}
	
	\begin{itemize}
	\item \emph{VCS} - version control system
	\end{itemize}
	
\begin{block}{Basic features}
	\begin{itemize}
		\item Keeps track of changes to our code
		\item Facilitates collaboration in software and documentation development
	\end{itemize}
\end{block}
	
\begin{block}{Additional requirements}
	\begin{itemize}
		\item Work offline
		\item Support a distributed, decentralized workflow
		\item Compatibility with existing protocols (e.g. ssh, http)
		\item Data integrity protection
		\item Efficiency
	\end{itemize}
\end{block}

\end{frame}

% -----------------------------------------------------------------------------

\begin{frame}

\frametitle{About git}
	
\begin{itemize}
	\item Developed in 2005 by Linus Torvalds (in a single weekend!) to support Linux kernel development
	\item Very powerful, has a reputation of being hard to learn
	\begin{itemize}
	\item We hope we will convince you otherwise :)
	\end{itemize}
	\item Supports different workflows
	\begin{itemize}
	\item We'll be using the \href{https://guides.github.com/introduction/flow/}{GitHub flow} (more or less)	
	\end{itemize}
\end{itemize}

\begin{block}{Meaning of the name \emph{git}}
\small
\emph{git} means "unpleasant person" in British slang. Linus: "I'm an egotistical bastard, and I name all my projects after myself."

\medskip
From the readme:
	 \begin{itemize}
	 \item "global information tracker": you're in a good mood, and it actually works for you. Angels sing, and a light suddenly fills the room.
	 \item "g*dd*mn idiotic truckload of sh*t": when it breaks
	 \end{itemize}
	 
\end{block}

\end{frame}

% -----------------------------------------------------------------------------

\begin{frame}

\frametitle{Basic terms: repository and commit}
\begin{block}{Repository (repo)}
	A place where your work is kept. It contains your code and its complete history, stored as a collection of commits.
\end{block}

\begin{block}{Commit}
	A basic unit of work in a project. Contains a \textbf{snapshot} of the complete project, a reference to a previous snapshot (the \textit{parent commit}), the commit message -- a textual description of the changes in the commit (with respect to the parent commit), the commit author name, and the commit date.
\end{block}	
\end{frame}

\begin{frame}

\frametitle{Commit ID (SHA)}
\begin{block}{Commit ID}

	\smallskip Every commit is marked with an alphanumeric identifier (SHA-1 hash) generated from the above information, which is used to uniquely identify the commit. \\For example: \textcolor{Maroon}{\texttt{bdfa760c07d8f621ff603a2dc5d6de810cd62e88}}
\smallskip

You can also use a prefix of the identifier to refer to this commit, usually 5 or 7 characters long, e.g. \textcolor{Maroon}{\texttt{bdfa760}}.
\end{block}
\end{frame}

% -----------------------------------------------------------------------------

\begin{frame}

\frametitle{Basic terms: branch, master and head}

\begin{block}{Branch, master and head}
A \alert{branch} is simply a pointer to a commit. \alert{Main} is usually the name of the main branch (but does not have to be). \alert{HEAD} is a special pointer to the currently checked out commit. \\
\smallskip
The commits with their parent-child relationships form a directed acyclic graph (DAG).
\end{block}

\begin{figure}
\includegraphics[scale=0.26]{branches}
\end{figure}

\end{frame}

% -----------------------------------------------------------------------------

\begin{frame}

\frametitle{Getting started on Github}

To begin working on a project using Github, you have to create a repository. There are three ways to do this:


\begin{itemize}
	\item Create an empty repository
	\begin{itemize}
	\item When starting from scratch
	\end{itemize}	
	
	\medskip
	\item Fork another repository
	\begin{itemize}
	\item When you wish to improve and build upon another repository
	\item This creates your personal copy of the repository in which you can make your own commits
	\item Common in open source projects with many contributors
	\end{itemize}
	
	\medskip
	\item Join an existing repository
	\begin{itemize}
	\item The owner must add you to the repository
	\item The most convenient collaboration mode for small and medium sized teams
	\end{itemize}
\end{itemize}

\end{frame}

% -----------------------------------------------------------------------------

\begin{frame}[fragile]

\frametitle{Adding a collaborator}

\begin{block}{Task A [Mirko]: Repository owner}

	Add Slavko as a collaborator to your repository through the Web UI with \texttt{write} access.
\end{block}

\begin{block}{Task B [Slavko]: Accept the collaboration invitation}
Check your mail, or the list of notifications. Accept the collaboration invitation.
\end{block}

\end{frame}

% -----------------------------------------------------------------------------

\begin{frame}[fragile]

\frametitle{Cloning the repository}

\begin{block}{Cloning}
Cloning creates a local copy of the repository, which includes all the commits in the repository -- the whole history.
\begin{minted}{console}
> git clone git@github.com:<Mirko>/<repo>.git
> cd <path to repo>
\end{minted}
\end{block}

\begin{block}{Task D [Slavko]}
Clone Mirko's repository onto your computer and \texttt{cd} into it.
\begin{minted}{console}
> git clone git@github.com:<Mirko>/<repo>.git
> cd <path to repo>
\end{minted}
\end{block}

\begin{itemize}
    \item NB -- if you committed something, it is in your local repo, (almost) permanently. Do not make it a habit to delete your local repositories -- if you "lose" a commit, it can be recovered (by its ID), even if you have deleted the branch!
\end{itemize}
	
\end{frame}

% -----------------------------------------------------------------------------

\begin{frame}

\frametitle{Visualizing your repository}

\begin{itemize}
	\item The \texttt{gitg} tool can visualize your repository
	\item GitHub provides similar functionality with the Graphs $\rightarrow$ Network menu.
\end{itemize}

\begin{block}{Task Q [Mirko, Slavko]}
	Visualize your repository with \texttt{gitg} and on GitHub. Notice how the history is not linear. 
\end{block}

\end{frame}

% -----------------------------------------------------------------------------

\begin{frame}[fragile]

\frametitle{Repository structure}

What is contained in the repository directory?
	
\begin{minted}{console}
> cd <path to repo>
> ls -la
> git status
\end{minted}
	
\begin{itemize}
	\item The \textit{working tree} -- current version of files ("checked out")
	\item The hidden \texttt{.git} folder which contains repository metadata \\ (all the commits)
	\item The \texttt{git status} command provides an overview of what is going on in the git repository.
\end{itemize}
\begin{minted}{console}
\end{minted}
	
\end{frame}

% -----------------------------------------------------------------------------

\begin{frame}
	\frametitle{Visualizing git operation}
	
	\begin{figure}
		\includegraphics[scale=0.4]{git-transport}
	\end{figure}
\end{frame}

% -----------------------------------------------------------------------------


\begin{frame}[fragile]
	\frametitle{Your first branch}
	
	\begin{block}{Task E [Mirko, Slavko]}
	Create a branch, off of the \texttt{main} branch:

	\begin{minted}{console}
> git checkout master
> git checkout -b <your name>-git-tutorial
> git status
> git branch -v
> gitg&
	\end{minted}
	
	\end{block}
	
\end{frame}

% -----------------------------------------------------------------------------


\begin{frame}[fragile]

\frametitle{Branching - recap}

\begin{itemize}
\item A \texttt{branch} is simply a "pointer" to a commit
\item \texttt{git checkout <branch>} updates the files in the working tree to match the specified branch
\item \texttt{git checkout -b <new branch>} creates a new branch at the current commit
\item You can use \texttt{gitg} to visualize your branches
\item Use descriptive branch names, separate words with dashes
\item Prefix "private" branches with your name
\item Branch early and often (but do not push every branch, and clean up regularly!)
\end{itemize}	
	
\end{frame}


% -----------------------------------------------------------------------------


\begin{frame}[fragile]
	\frametitle{Your first commit}
	
	\begin{block}{Task E [Mirko, Slavko]}
	Open \texttt{README.md} and change some lines, then save the file:

Run the following, and observe what is happening.

	\begin{minted}{console}
> git status
> git diff
> git add README.md
> git status
> git gui&
> git commit -m "<details about the change>"
> git status
> git log
	\end{minted}
	
	\end{block}
	
\end{frame}

% -----------------------------------------------------------------------------


\begin{frame}[fragile]

\frametitle{Making a commit - recap}
	
\begin{enumerate}
	\item Make a change in the working tree. For example:
	\begin{itemize}
	\item Edit a file
	\item Create a file
	\item Delete a file 
	\item Move a file (git considers moving as deleting + creating a new file)
	\end{itemize}
	\item \texttt {git add} the change to the \textit{staging area} ("index")
	\item Perform a final inspection of the staged changes
	\begin{itemize}
	\item \texttt{git gui} is handy for this
	\item The staged changes should make a logically grouped set of changes
	\item Feel free to make as many commits as you like
	\item "Commit early and often"
	\end{itemize}
	\item \texttt{git commit} your changes
	\begin{itemize}
	\item  "50/72" rule - the title of the commit description should be {\raise.17ex\hbox{$\scriptstyle\sim$}}50 characters, the body should be wrapped to 72 characters
	\item Use the imperative form as the first word in the title -- e.g. \\"Add prime checking", "Implement saving to file", "Fix broken build"
	\end{itemize}
\end{enumerate}
	
\end{frame}

% -----------------------------------------------------------------------------

\begin{frame}

\frametitle{Managing staged changes}

	\begin{block}{Task F [Mirko, Slavko]}
	Make some more changes to some file(s) in your working tree.
	\end{block}
	\begin{itemize}	
	
	\item To get a list of staged and unstaged files, run \texttt{git status}.
	\item To unstage all changes, run \texttt{git reset}.
	\item To view the \textbf{unstaged} changes in the \textit{diff} format, open \texttt{git gui}, 
	\\or run \texttt{git diff}

	\item To view the \textbf{staged} changes in the \textit{diff} format, you can also use \texttt{git gui}, or you can run \texttt{git diff --cached}
	
	\item You can finely tune what goes into the commit by staging/unstaging individual lines using \texttt{git gui}.
	(The shortcut for (un)staging all changes in a file is Ctrl-U)
	\end{itemize}
	
\end{frame}


% -----------------------------------------------------------------------------

\begin{frame}[fragile]
	\frametitle{Discarding unstaged changes}
	
	\begin{block}{Task H [Mirko, Slavko]}
	\begin{itemize}
	\item Delete a big chunk of text from an important file. Do not add or commit.
	
	\item To find out what is going on, use \texttt{git status} and \texttt{git diff}.
	\medskip	
    \item Discard the changes by retrieving ("checking out") the last commited version of the file (can also be a directory) using the \texttt{checkout} command:

	\begin{minted}{console}
> git checkout <file>
	\end{minted}
	
	Note that the checked out version will contain the staged changes.
	\end{itemize}	
	\end{block}
\end{frame}
% -----------------------------------------------------------------------------

%\begin{frame}[fragile]
%
%\frametitle{Discarding many unwanted changes}
%
%	\begin{itemize}
%	\item If you have made many changes in your working tree, spanning several files, you can discard all of them at once (this also includes the staged changes): 
%	\begin{minted}{console}
%> git reset --hard
%	\end{minted}
% 
%	\item This will restore all tracked files in the working tree to the most recently committed version (i.e. the \texttt{HEAD} commit).
%	\end{itemize}
%
%\begin{block}{Task I [Mirko, Slavko]: Discarding several unwanted changes}
%	\begin{itemize}
%	\item Delete several files from the cloned example repository. Do not commit the changes.
%	\item Check the output of \texttt{git status}.
%	\item Discard all changes as shown above.
%	\end{itemize}
%\end{block}
%
%\end{frame}

% -----------------------------------------------------------------------------


\begin{frame}

\frametitle{Going back in time 1/3}

	\begin{itemize}	
	\item A big point in using a version control system such as git is the ability to retrieve older versions of files in the project. 
	\item Note that writing good commit messages is important, because it will make identifying the right version much easier.
	\item In the workflow we have described, the procedure for undoing mistakes is making additional commits which fix these mistakes.
	\end{itemize}
\end{frame}

% -----------------------------------------------------------------------------

\begin{frame}[fragile]

\frametitle{Going back in time 2/3}

\begin{block}{Task I: Checking out an earlier commit [Mirko, Slavko]}
	\begin{itemize}
	\item Identify an interesting commit in your history by either:
	\begin{itemize}
	\item running \texttt{git log} (add \texttt{--oneline} for an abbreviated list of commits),
	\item using the \texttt{gitg} tool. 
	\end{itemize}
	\item Check out that commit into your working tree
	\begin{minted}{console}
> git checkout <commit ID>
	\end{minted}
	\item This will bring you in a so-called \emph{detached head} state (sounds more dangerous than it acually is :)
	\item This is useful in situations where you want to build a previous version of your code that you know is working
	\item Bring \texttt{HEAD} back to the tip of your tutorial branch
	\begin{minted}{console}
> git checkout <your name>-git-tutorial
	\end{minted}
	\end{itemize}
\end{block}

\end{frame}

% -----------------------------------------------------------------------------

\begin{frame}[fragile]

\frametitle{Going back in time 3/3}

Sometimes, you want to retrieve a previously committed version of a specific file.

	\begin{block}{Task J [Mirko, Slavko]}	
	\begin{itemize}	
	\item Identify a specific file version in your commit history.
	\item Perform the retrieval by running:
	\begin{minted}{console}
> git checkout <commit ID> <file>
	\end{minted}
	\item Git will place the checked out version of the file in the \emph{staging area}.
	\item How do you make further tweaks to the checked out version?
	\item How do you discard this version?
	\end{itemize}	
	
	\end{block}
\end{frame}

% -----------------------------------------------------------------------------

\begin{frame}[fragile]
	\frametitle{Pushing the changes to the remote repository}
	
	\begin{itemize}
	\item \texttt{commit} only saves changes \alert{locally}!
	\item The command \texttt{git push} is used to upload the commits you have made to a branch in a remote repository (also known as just \textit{remote}).
	\item When you are pushing a local branch for the first time, tell git that to track the remote branch with your local branch:
	\begin{minted}{console}
> git push -u origin <branch name>
	\end{minted}
	\item On subsequent pushes you can simply call
	\begin{minted}{console}
> git push
	\end{minted}
	\end{itemize}

	\begin{block}{Task K [Mirko, Slavko]}
	Push the commits you have made to Mirko's repository.
	\end{block}

\end{frame}

% -----------------------------------------------------------------------------

\begin{frame}
	\frametitle{Binary files}
	
	\begin{block}{Difference between text and binary files}
	\begin{itemize}
	\item Changes to files are stored incrementally, as commit diffs, which is very space-efficient for text-files.
    \item For binary files, most of the time, the majority of the file is changed (e.g. when you edit a picture, or recompile an executable), which effectively means that the commit \textbf{contains a complete copy of the new version}.
    \begin{itemize}
    \item The old version \textbf{still persists} in the old commits, \textbf{even if you remove the file}, because git keeps the whole history.
    \end{itemize}
    \end{itemize}
	\end{block}
	
	\begin{block}{How to handle binary files}
	Storing binary files (e.g. graphics) is acceptable if they are small and change infrequently. Otherwise, create a README file with instructions for downloading the files, or a script which downloads them into place.
	\end{block}
\end{frame}

% -----------------------------------------------------------------------------

\begin{frame}[fragile]
	\frametitle{Binary files - build output}
	
	\begin{block}{Build output}
	\begin{itemize}
	\item \alert{Never} commit build output (even if it is text, e.g. documentation)! It can waste \alert{huge} amounts of space and it will create unnecessary conflicts
	\item Add the appropriate entries to \texttt{.gitignore} so that git ignores \\ the build output
	\begin{itemize}
	\item This will also keep \texttt{git gui} and the output of \texttt{git status}\\ free from clutter related to build files
	\end{itemize}
	\end{itemize}
	\end{block}
\end{frame}

% -----------------------------------------------------------------------------

%\begin{frame}[fragile]
%
%\frametitle{Reverting previous commits}
%
%	\begin{itemize}
%	\item Another benefit of splitting the work in different commits is the ability to undo them using \texttt{git revert}.
%	\item To revert the offending commit:
%	\begin{minted}{console}
%> git revert <commit ID>
%	\end{minted}
%	\item Reverting generates a \textit{revert commit}, which has the exactly opposite (inverse) set of changes.
%	\end{itemize}
%
%	\begin{block}{Task N [Slavko]: Slavko gets his revenge}
%	\begin{itemize}
%	\item Revert Mirko's annoying commit.
%
%	\end{itemize}
%	\end{block}
%	
%\end{frame}

% -----------------------------------------------------------------------------

%\begin{frame}[fragile]
%
%\frametitle{Amending the last commit}
%
%In the special case when you want to add aditional changes to the \textit{last commit} that you have made, \textbf{and if you have not yet pushed that commit}:
%
%	\begin{enumerate}
%	\item Stage the additional changes using \texttt{git add}
%	\item Inspect the changes, as previously described
%	\item Run \texttt{git commit --amend}
%	\begin{itemize}
%	\item In case you only wish to amend the commit description, just run \texttt{git commit --amend}, without staging any changes.
%	\end{itemize}
%	\end{enumerate}
%
%	\begin{block}{Task O [Slavko]}
%	Amend the description of the revert commit that you have just made: \texttt{Revert Mirko's rudeness}. Push the revert commit.
%
%\end{block}
%
%	\begin{block}{Task P [Mirko]}
%	Pull Slavko's changes.	
%	\end{block}
%
%
%\end{frame}

% -----------------------------------------------------------------------------