\section{Collaborating with others}

% -----------------------------------------------------------------------------


\begin{frame}[fragile]
	\frametitle{Getting changes from the remote repository}
	
	Getting remote changes is a two-step process:
	\begin{enumerate}
		\item \texttt{fetch} the changes from the remote repository
	\begin{minted}{console}
> git fetch
> git status
> git diff <your branch> origin/<other branch>
	\end{minted}
		\item \texttt{merge} the changes into your local branch
	\begin{minted}{console}
> git merge origin/<branch>
	\end{minted}
	\end{enumerate}
	
	\begin{block}{Understanding \texttt{diff}}
	\texttt{git diff a b} shows changes that need to be applied to \texttt{a} to make it the same as \texttt{b}. \texttt{a} and \texttt{b} are references to any two commits.
	\end{block}
	
\end{frame}

% -----------------------------------------------------------------------------

\begin{frame}[fragile]
	\frametitle{Syncing your code: merging}
	\begin{itemize}
	\item When you call \texttt{merge}, git adds a special commit known as a \textit{merge commit} to your local branch
	\item The merge commit has two parent commits: your local previous head commit, and the other side's commit.
	\end{itemize}
	\begin{block}{\texttt{fetch}+\texttt{merge}=\texttt{pull}?}
	\begin{itemize}	
	\small
	\item \texttt{pull} will do \texttt{fetch} and \texttt{merge} in a single step
	\item If the remote version has changed, and you have made local commits before pulling the remote changes, you might get yourself into trouble.\footnote{A really nice article advocating the use of \texttt{fetch}+\texttt{merge} can be found on \href{http://longair.net/blog/2009/04/16/git-fetch-and-merge/}{Mark's Blog}} 
	\item When you are getting back to work on a branch that you share with someone else, always \texttt{fetch}first to avoid unpleasant surprises.
	\end{itemize}
	\end{block}
\end{frame}

% -----------------------------------------------------------------------------

\begin{frame}[fragile]
	\frametitle{Merge conflicts}

	\begin{block}{Task O [Slavko]}
	\begin{itemize}
	\item Merge the remote changes
	\item Resolve the conflicts (if any)
	\item Push your changes
	\item (After Mirko completes his task) Fetch the new \texttt{master}
	\end{itemize}
	\end{block}

	\begin{block}{Task P [Mirko]}
	\begin{itemize}

	\item Fetch Slavko's changes
	\item Merge Slavko's changes into your branch
	\item Use \texttt{gitg} to examine what happened
	\end{itemize}
	\end{block}

\end{frame}

% -----------------------------------------------------------------------------
\begin{frame}[fragile]
	\frametitle{Resolving merge conflicts}
	
	After merging, some files might end up in a conflicted state:
	\begin{minted}{console}
> git merge origin/<branch>
> git status
> git gui
	\end{minted}	
	
	Conflict markers inside the file:
	\begin{minted}{diff}
<<<<<<< HEAD
Code from the checked out branch ("local" in git gui) 
=======
Code from the other branch ("remote" in git gui)
>>>>>>> origin/<branch name>
	\end{minted}

	To resolve the conflict, manually edit the file, mark resolution with \texttt{git add}, commit and push:
	\begin{minted}{console}
> git add <file name>
> git commit -m "<commit message>"
> git push origin <branch>
	\end{minted}
	
\end{frame}

% -----------------------------------------------------------------------------

\begin{frame}[fragile]

\frametitle{Visualizing merges}

\begin{multicols}{2}
	\begin{figure}
		\includegraphics[scale=0.5]{3-way-merge-before}
	\end{figure}
	\begin{figure}
		\includegraphics[scale=0.5]{3-way-merge-after}
	\end{figure}
\end{multicols}

\end{frame}

% -----------------------------------------------------------------------------

\begin{frame}[fragile]

\frametitle{Reverting previous commits}

	\begin{itemize}
	\item Another benefit of splitting the work in different commits is the ability to undo them using \texttt{git revert}.
	\item To revert the offending commit:
	\begin{minted}{console}
> git revert <commit ID>
	\end{minted}
	\item Reverting generates a \textit{revert commit}, which has the exactly opposite (inverse) set of changes.
	\end{itemize}

	\begin{block}{Task Q [Mirko]: Mirko gets his revenge}
	\begin{itemize}
	\item Revert Slavko's annoying commit.
	\item Use \texttt{gitg} to examine what happened
	\item Merge your branch into the \texttt{master} branch
    \item Push your \texttt{master} branch to \texttt{origin}
	\end{itemize}
	\end{block}
\end{frame}

% -----------------------------------------------------------------------------

\begin{frame}[fragile]

\frametitle{Cleaning up local branches}

Branches are a powerful and useful tool. However, to be used effectively, we must maintain them regularly. Generally, there are two types of branches
\begin{itemize}
	\item Short-lived branches (for bugfixes and features)
	\item Long-lived branches
\end{itemize}

After merging a short-lived branch, perform cleanup by deleting it:
\begin{minted}{console}
> git branch -d <branch name>
\end{minted}

To delete an unmerged branch, you have to use the \texttt{-D} option, but be careful with this option, as the commits it was pointing to will become difficult to reach.

\end{frame}


% -----------------------------------------------------------------------------

\begin{frame}[fragile]

\frametitle{Cleaning up remote branches}

\begin{block}{Deleting remote branches}
	\begin{itemize}
	\item Can be done from the Github interface;
	\\ run \texttt{git fetch --prune} afterwards to clean them up in your local copy of the remote repository
	\item To delete a remote branch using the command-line interface, run \texttt{git push --delete <remote> <branch\_name>}
	\end{itemize}
\end{block}	

\begin{block}{Task R [Mirko, Slavko]: Clean up your branches}
	Delete both the local and remote branches.
\end{block}	
\end{frame}

% -----------------------------------------------------------------------------

\begin{frame}[fragile]
	
\frametitle{Issues and pull (merge) requests 1/2}

\emph{Pull requests} are an efficient and transparent code review mechanism. GitLab calls them \emph{Merge requests}.

\begin{block}{Task S [Mirko]: Create an Issue}
Examine the demo application and create an issue through the GitHub Web UI.
\end{block}

\begin{block}{Task T [Slavko]: Merge request}
Implement the requested fix in a new branch. Commit, and do not forget to \emph{reference the issue} in the commit message. Create a Merge request.
\end{block}	

\begin{block}{Task U [Mirko, Slavko]: Code review}
Perform code review, merge the MR and delete the branch.
\end{block}


\end{frame}

% -----------------------------------------------------------------------------

\begin{frame}[fragile]

\frametitle{Issues and pull (merge) requests 2/2}

\emph{Pull requests} are an efficient and transparent code review mechanism. GitLab calls them \emph{Merge requests}.

\begin{block}{Task V [Slavko]: Create an Issue}
Ask for a \texttt{square} command to be implemented for the demo shell program.
\end{block}

\begin{block}{Task W [Mirko]: Merge request}
Implement the requested feature in a new branch. Commit, and do not forget to \emph{reference the issue} in the commit message. Create a Merge request.
\end{block}

\begin{block}{Task X [Mirko, Slavko]: Code review}
Perform code review, merge the MR and delete the branch.
\end{block}	


\end{frame}

% -----------------------------------------------------------------------------

%\begin{frame}[fragile]
%	
%\frametitle{Working with multiple remotes}
%	
%\begin{block}{Why would I need multiple remotes?}
%	\begin{itemize}
%	\item We can get code changes from any repository, not just our remote repository (which is automatically set up as a remote named \texttt{origin} when cloning)
%	\item A typical example is getting changes from the \textit{upstream} repository, i.e., the repository that we forked.
%	\end{itemize}
%\end{block}	
%
%Listing and adding remotes:
%\begin{minted}{console}
%> git remote add <remote_name> git@<hostname>:<path to repo>	
%> git remote -v	
%\end{minted}
%	
%We can now work with the new remote in the same way as with \texttt{origin} (except we can't push into it!), e.g.:
%\begin{minted}{console}
%> git fetch <remote_name>
%> git merge <remote_name>/<branch>
%\end{minted}
%	
%\end{frame}

% -----------------------------------------------------------------------------

%\begin{frame}
%
%\frametitle{Merging upstream changes}
%
%\begin{block}{Task: Merge upstream changes}
%	\begin{itemize}
%	\item Set up a remote called \texttt{upstream} pointing to \href{https://github.com/larics/git-tutorial-code.git}{the original repository you forked}
%	\item \texttt{fetch} the \texttt{upstream} repository and compare its \texttt{master} branch with your \texttt{master} branch.
%	\item Optional: for experimenting how the upstream changes will play with your changes, create a new branch.
%	\item Merge the upstream \texttt{master} branch into your \texttt{master} branch (or first into your experimental branch, and then merge your experimental branch into your \texttt{master}).
%	\end{itemize}
%\end{block}
%
%\end{frame}

% -----------------------------------------------------------------------------
